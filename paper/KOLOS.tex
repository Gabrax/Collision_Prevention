\documentclass[12pt]{article}
\usepackage[utf8]{inputenc}
\usepackage[T1]{fontenc}
\usepackage[polish]{babel}
\usepackage{geometry}
\usepackage{parskip}
\usepackage{titlesec}
\usepackage{lmodern}
\usepackage{multicol}
\usepackage{tocloft}
\usepackage{setspace}
\usepackage{csquotes}

\geometry{a4paper, margin=2.5cm}
\titleformat{\section}{\normalfont\Large\bfseries}{\thesection}{1em}{}
\titleformat{\subsection}{\normalfont\large\bfseries}{\thesubsection}{1em}{}
\renewcommand{\cftsecleader}{\cftdotfill{\cftdotsep}}

\begin{document}

\tableofcontents

\newpage

\section{LATEX}

\subsubsection{System składu}

\TeX{} to system składu drukarskiego, dostępny w każdym używanym współcześnie systemie operacyjnym. Cechuje go m.in.: wysoka jakość składu, możliwość dopasowywania do specjalizowanych zadań, łatwa współpraca z innymi aplikacjami, przenośność (na każdej platformie systemowej działa identycznie). Rozpowszechniany jest na licencji typu Open Source.

Za pomocą dowolnego edytora lub w inny sposób tworzony jest plik tekstowy, zawierający treść wymieszaną z poleceniami języka \TeX{}. Plik ten jest interpretowany przez program \texttt{tex} i tworzony jest plik pośredni opisujący dokument wynikowy strona po stronie (plik \texttt{.dvi}). Plik \texttt{.dvi} zawiera tylko opis, nie zawiera wielu zasobów niezbędnych do reprodukcji na urządzeniu wyjściowym, takich jak fonty (kształty znaków) i pliki graficzne. Sterownik zamienia plik \texttt{.dvi}, korzystając z niezbędnych zasobów na dokument docelowy możliwy do reprodukcji na określonym urządzeniu.

Informacje zamieszczone w tym dokumencie zaczerpnięte zostały ze strony \texttt{www.gust.org.pl}

\section{Odstępy}

\subsection{Odstępy między liniami, podział na akapity i wyrównania}

Ile równań niezależnych, ile jest szeregów zbieżnych, ile całek niewłaściwych,\\
ile na płaszczyźnie krzywych, ile funkcji kwadratowych, co nie mają miejsc zerowych,\\
ile krzywe mają siecznych, ile jest układów sprzecznych, ile różnych jest symetrii,\\
ile twierdzeń w geometrii, ile przestrzeń ma wektorów, co nie tworzą pustych zbiorów,\\
Tyle szczęścia i radości\\
w Twoim domu,\\
razem z Gwiazdką, niech zagości.

\begin{flushright}
Ból jest świętym aniołem – on to bardziej aniżeli wszystkie inne radości tego świata sprawił, iż wielu osiągnęło dojrzałość.

\hfill -- A. Stifter

Trzeba ukochać życie, aby je dobrze przeżyć – ukochać je nie tylko w wielkich porywach i szczytnych uniesieniach, ale w powszednim wysiłku i codziennym mozole.

\hfill -- H. Bordeaux

W dzisiejszych czasach nie można znać wszystkiego na pamięć.\\
Człowiek wykształcony nie jest tym,\\
który wszystko wie;\\
jest on takim człowiekiem, który wie, gdzie należy szukać wiadomości.

\hfill -- J. Arsac
\end{flushright}

\begin{center}
Przechodź spokojnie, przez hałas i pośpiech i pamiętaj jaki spokój, można znaleźć w ciszy.

O ile to możliwe, bez wyrzekania się siebie, bądź na dobrej stopie ze wszystkimi.

Wypowiadaj swą prawdę jasno i spokojnie i wysłuchuj innych, nawet tępych i nieświadomych, oni też mają swoją opowieść.(...)

Anonimowy tekst z 1692 znaleziony w starym kościele św. Pawła w Baltimore
\end{center}

\subsection{Odstępy między literami i wyrazami}

\begin{quote}
\enquote{W tajemnicy każdego człowieka\\
istnieje wewnętrzny krajobraz:\\
z nietkniętymi równinami,\\
z wąwozami milczenia,\\
z niedostępnymi górami,\\
z ukrytymi ogrodami.}

\hfill -- Antoine de Saint-Exupéry
\end{quote}

\begin{quote}
Spotykamy trzy \hfill rodzaje ludzi:\\
Tych, \dotfill którzy mnie potrzebują\\
tych, \leavevmode\leaders\hrule height 0.4pt\hfill\kern0pt których ja potrzebuję\\
i\hspace{1em}przyjaciół, \dotfill\ z którymi lubię\hspace{1em}%
\rule{1cm}{0.4pt} \dotfill\ przebywać. \dotfill

\hfill -- Ewa Lewandowska
\end{quote}

\end{document}

